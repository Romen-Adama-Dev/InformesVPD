% ------------------------------------------------------------------------------
% Virtualizacion y Procesamiento Distribuido
% by Romen Adama Caetano Ramirez
% University of Las Palmas de Gran Canaria 
% Chapter Introduccion
% ------------------------------------------------------------------------------
\chapter{Introduccion} % enter the name of the chapter here

\section{Resumen} % enter the name of the section here

\subsection*{Creación y configuración de una máquina virtual}
Esta actividad tiene dos objetivos principales: instalar la plataforma de virtualización KVM y crear máquinas virtuales utilizando esta plataforma. 
Para lograr esto, se recomienda consultar dos fuentes bibliográficas de Red Hat: la Guía de Inicio para la Virtualización, que proporciona información básica sobre virtualización y los componentes principales de la plataforma de virtualización de Red Hat.

Ademas se recomienda consultar la Guía de Implementación y Administración de la Virtualización, que describe la instalación de la plataforma de virtualización Red Hat en un sistema anfitrión Linux Red Hat Enterprise, la instalación de máquinas virtuales, y la administración del sistema anfitrión y las máquinas virtuales utilizando diferentes recursos de la plataforma de virtualización de Red Hat. La segunda fuente es la base de esta guía.


\section{Parámetros	de	la	máquina	virtual} % enter the name of the section here

Para la creación de máquina virtual vamos a usar la utilidad virt-manager. En ella elegimos crear una nueva máquina virtual. A continuación hay que proporcionar los siguientes parámetros. 

\begin{enumerate}
    \item Nombre de la máquina virtual y medio de instalación del S.O. 
    \item Sistema operativo a instalar en la máquina virtual 
    \item Cantidad de memoria (1GB) y número de procesadores (1)
    \item Características del disco de la máquina virtual: tamaño, estático o dinámico y ubicación del archivo en el anfitrión que lo simula (5 GB dinámico). 
    \item Especificación de características avanzadas de la máquina virtual: Interfaz de red, MAC, tipo de virtualización y arquitectura del procesador (1 interfaz de red en modo NAT). 
    \item Paso final: pantalla que indica el avance en la ejecución de la orden de creación de la máquina virtual especificada.
\end{enumerate}


\section{Algunas consideraciones adicionales}
\begin{enumerate}
    \item Con	 la	 utilidad	 virt-manager se	 pueden	 consultar	 las	 características	
hardware	 de	 la	 máquina	 virtual.	 Comprobar	 que	 se	 corresponde	 las	
características	mostradas	con	las	suministradas.
    \item Observar	 que	 se	 puede	 modificar	 el	 hardware,	 modificando	 las	
características	del	hardware	que	tiene	o	añadiendo	hardware	virtual.
    \item Averiguar	a	qué	red	está	conectada	la	máquina	virtual.
    \item Comprobar que	hay	conectividad	entre	la	máquina	virtual	y	el	anfitrión.
    \item Comprobar que	hay	conectividad	entre	la	máquina	virtual	y	el	exterior.
\end{enumerate}


\section{Objetivos}
El objetivo final de esta práctica es aprender a instalar y utilizar la plataforma de virtualización KVM para crear y administrar máquinas virtuales en un sistema anfitrión Linux Red Hat Enterprise. 

Se debe seguir una serie de pasos para instalar KVM y crear una máquina virtual con los parámetros necesarios, incluyendo la elección del sistema operativo, la cantidad de memoria y procesadores, el tamaño y tipo de disco, y las características avanzadas de la máquina virtual como la interfaz de red y la arquitectura del procesador. 

También se deben realizar algunas comprobaciones adicionales, como verificar las características de hardware de la máquina virtual y la conectividad con el anfitrión y el exterior. 

En resumen, la práctica tiene como objetivo proporcionar una comprensión práctica de la virtualización y la administración de máquinas virtuales utilizando la plataforma KVM.